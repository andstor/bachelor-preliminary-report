\section{Progress plan}

\subsection{Master plan}
\label{sec:master-plan}
Figure~\ref{fig:gantt-diagram-progress-plan} presents a gantt diagram for the planned time scheduling. This includes all activities listed in section \ref{sec:primary-activities}. These activities primarily include writing and software development. Activities A1 is concerned about writing the preliminary report and the thesis. Activity A2, A3, A4, A5, and A6 are concerned with the development of various software systems, where each activity is a confined project. A7 is concerned with automation of various tasks in many of the software projects.

{
\begin{figure}[h]
\thispagestyle{empty}
%\begin{landscape}
	\clearpage
	\newgeometry{margin=2.5cm, top=20mm, bottom=35mm}
\definecolor{foobarblue}{RGB}{0,153,255}
\definecolor{foobaryellow}{RGB}{234,187,0}
\newganttchartelement{mainbar}{
    mainbar/.style={
        draw=foobarblue!10!black,
        fill=foobarblue!10
    },
    mainbar label font=\bfseries
}

\hspace{-2.25cm}
\noindent
\begin{ganttchart}[
    hgrid,
    vgrid={*{6}{draw=none},{dotted}},
    time slot format=isodate,
    time slot unit=day,
    calendar week text = {W\currentweek{}},
    vrule label font=\tiny,         
    title label font=\tiny,
    %bar label font=\tiny,
    %milestone label font=\tiny,
    %group label font=\bfseries\small,
    bar height = 0.4,
    milestone left shift =-1,
    milestone right shift =1,
    x unit=0.105cm,
    y unit title=.6cm,
    y unit chart=0.75cm,
    link bulge = 2,
    link/.style={-to, rounded corners = 2pt}
    ]{2020-01-06}{2020-05-24}
    %labels
    \gantttitlecalendar{year, month=name, week=2} \\

    %tasks
    \ganttgroup[name=A1]{Writing}{2020-01-08}{2020-05-20} \\
    \ganttbar[name=A11]{Preliminary report}{2020-01-08}{2020-01-31} \\
    \ganttmilestone[name=M11]{Preliminary report completed}{2020-01-31} \\
    \ganttbar[name=A12]{Bachelor's thesis}{2020-02-03}{2020-05-20} \\
    \ganttmilestone[name=M12]{Bachelor's thesis completed}{2020-05-20} \\

    % three-voxel-loader
    \ganttmainbar[name=A3]{three-voxel-loader}{2020-02-03}{2020-02-14} \\
    \ganttmilestone[name=M3]{three-voxel-loader completed}{2020-02-14} \\
 
    % jsdoc-action
    \ganttmainbar[name=A6]{jsdoc-action}{2020-02-17}{2020-02-21} \\
    \ganttmilestone[name=M6]{jsdoc-action completed}{2020-02-21} \\

    % Automation
    \ganttmainbar[name=A7]{Automation}{2020-02-17}{2020-02-21} \\
    \ganttmilestone[name=M7]{Automation implemented}{2020-02-21} \\

    % voxelizer
    \ganttgroup[name=A2]{voxelizer}{2020-02-24}{2020-04-15} \\
    \ganttbar[name=A21]{Core improvements}{2020-02-24}{2020-02-28} \\
    \ganttbar[name=A22]{Algorithm improvements}{2020-03-02}{2020-03-13} \\
    \ganttbar[name=A23]{Texture support}{2020-03-16}{2020-03-20} \\
    \ganttbar[name=A24]{Extending 3D model file loading}{2020-03-23}{2020-03-27} \\
    \ganttbar[name=A25]{Extending data exporting}{2020-03-30}{2020-04-03} \\
    \ganttbar[name=A26]{Write tests}{2020-04-06}{2020-04-08} \\
    \ganttbar[name=A27]{Optimization}{2020-04-14}{2020-04-15} \\
    \ganttmilestone[name=M2]{voxelizer completed}{2020-04-15} \\
 
    % voxelizer-desktop
    \ganttgroup[name=A4]{voxelizer-desktop}{2020-04-16}{2020-05-07} \\
    \ganttbar[name=A41]{Core}{2020-04-16}{2020-04-22} \\
    \ganttbar[name=A42_1]{GUI}{2020-04-23}{2020-04-30}
    \ganttlinkedbar[name=A42_2]{}{2020-05-04}{2020-05-07} \\
    \ganttmilestone[name=M4]{voxelizer-desktop completed}{2020-05-07} \\

    % voxelizer-cli
    \ganttmainbar[name=A5]{voxelizer-cli}{2020-05-08}{2020-05-14} \\
    \ganttmilestone[name=M5]{voxelizer-cli completed}{2020-05-014} \\
    
    % relations
    % Writing
    \ganttlink{A11}{M11}
    \ganttlink[link type=dr]{M11}{A12}
    %\ganttlink{M11}{A12}
    \ganttlink{A12}{M12}
    % three-voxel-loader
    \ganttlink[link type=dr]{M11}{A3}
    \ganttlink{A3}{M3}
    % jsdoc-action
    \ganttlink[link type=dr]{M3}{A6}
    \ganttlink{A6}{M6}
    % Automation
    \ganttlink[link type=dr]{M3}{A7}
    \ganttlink{A7}{M7}
    % voxelizer
    \ganttlink{M7}{A21}
    \ganttlink{A21}{A22}
    \ganttlink{A22}{A23}
    \ganttlink{A23}{A24}
    \ganttlink{A24}{A25}
    \ganttlink{A25}{A26}
    \ganttlink{A26}{A27}
    \ganttlink{A27}{M2}
    % voxelizer-desktop
    \ganttlink{M2}{A41}
    \ganttlink{A41}{A42_1}
    \ganttlink{A42_2}{M4}
    % voxelizer-cli
    \ganttlink{M4}{A5}
    \ganttlink{A5}{M5}
    

\end{ganttchart}
\restoregeometry
	\caption{Gantt diagram of progress plan.}
	\label{fig:gantt-diagram-progress-plan}
\end{figure}
\clearpage
}

%%===================================
\subsection{Project control assets}
\label{sec:project-control-assets}
In order to keep the project on track, Jira \cite{jira} will be used. Jira is a project management tool developed by Atlassian, supporting a vast number of features such as issue tracking and project management. The main reason for choosing Jira over for example GitHub's solutions, is that Jira supports the agile methodology Scrum. For managing documents, minutes of meetings, UML diagrams, etc., Confluence \cite{confluence} will be used.

Since this project revolves around open source projects, Jira and Confluence will only be used for internal related work. For public usage, the GitHub issue tracker and wiki will be used. Any issues, bugs, or documentation of public interest shall be be placed on GitHub, instead of Jira and Confluence.

%%===================================
\subsection{Development assets}
For developing the various systems, the development tools listed in Table \ref{tab:development-tools} will be used.

\vspace{1em}

\begin{table}[H]
\centering
\caption{Development tools.}
\label{tab:development-tools}
\begin{tabular}{ |l@{\hspace*{1.5em}}|@{\hspace*{1.5em}}l| }
	\hline
	\textbf{Development tool} & \textbf{Description} \\
	\hline
	Visual Studio Code \cite{visual-studio-code} & Editor for writing and debugging code.\\
	\hline
	Blender \cite{blender} & 3D modeling software. \\
	\hline
	Git \cite{git} & Version control.\\
	\hline
	GitHub \cite{github} & Hosting of git repositories. \\
	\hline
	SourceTree \cite{sourcetree} & Git desktop client. \\
	\hline
\end{tabular}
\end{table}

%%===================================
\subsection{Internal control and evaluation}
At the end of each sprint, a review of the completed sprint will be conducted. A burndown chart will be generated for each sprint. This will help identifying if adjustments to the plan is necessary.

The requirements specification will serve as the primary criteria in order to decide whether a goal is completed or not. If a system is implemented but contains minor bugs, it will still be considered complete.
