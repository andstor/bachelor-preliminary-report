%This is chapter 5
%%=========================================
\chapter{Planned deviation management}

In the event of deviations from the current plans, both in terms of progress or content, several measures needs to be taken. If the deviations from the plan are of greater significance, the supervisor should be alerted. If the deviation is of lesser importance, it should be discussed with the supervisor at the regular meeting. One should then consider to change the planned approach.

Many of the planned systems builds upon one another. Therefore, if a task shows to be harder and more time consuming than first anticipated, it should consume time from tasks of lower priority. However, if a task exceeds its planned time scedule because of minor bugs, then these bugs should be properly documented and the task considered finished. These bugs should then be revisited at a later stage if there is time to spare. Since the systems are open source projects, these bugs might also be resolved by volunteers after this project is finished.
