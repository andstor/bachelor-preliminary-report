%This is chapter 1
%%=========================================

\chapter[Introduction]{Introduction}
This project aims to improving an already existing open source \cite{open-source} engine named Voxelizer \cite{voxelizer}. It is a cross-platform engine for conducting voxelization of 3D models, and is written in JavaScript.

The background for its creation was an assignment in a simulation course. The objective was to simulate diffusion using a cellular automaton. I wanted to do the simulation in the shape of a 3D object. Hence, I needed some way of constructing a volume representation out of a 3D model. Further, I also wanted to make the simulation with web technologies by making use of Three.js \cite{three.js}, an abstraction layer over WebGL \cite{webgl}. 

To the best of my knowledge, there was not any simple open source solution for this in JavaScript. I therefore decided to make a solution myself. The result was the open source project "Voxelizer". However, due to time constraints, the current state of the engine can only be considered a crude prototype. It has several issues, lacks important features and needs to be professionalized.

Alongside the engine, a desktop program and a CLI interface will be developed. These will be making use of the Voxelizer software, and will greatly simplify the use of it.